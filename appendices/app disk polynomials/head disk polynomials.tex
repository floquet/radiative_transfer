\chapter{Disk polynomials}

The disc polynomials defined Zernike are the unique complete set of polynomials orthogonal over the closed unit disc $\disc$ which have a discrete rotational invariance and an order-by-order correspondence with the Taylor monomials. They are complex-valued functions with dual indices $n$ and $m$, both of which are non-negative integers. The dominant index $n$ specifies the order and $m$ an angular frequency. These functions are naturally defined in polar coordinates where they are separable into radial and angular terms. The definition in the complex plane reveals which polynomials are real, analytic or complex. Usually Zernike's complex polynomials are resolved into pairs of real functions which express the real and imaginary components of a complex-valued function.

In our analysis we have used a subset of the polynomials defined by angular frequency $m=0$; these terms have a continuous rotational invariance. The lowest order terms are seen in figure \eqref{fig:Zernike cutaways}.

\begin{figure}[htbp] %  figure placement: here, top, bottom, or page
   \centering
   \includegraphics[ width = 5.75in ]{graphics/Zernike/"cutaways table"} 
   \caption{The first 12 functions in the sequence $Z_{0}^{0}(r)$, $Z_{2}^{0}(r)$, $Z_{4}^{0}(r)$ \dots .}
   \label{fig:Zernike cutaways}
\end{figure}

%%%%%%%%
\section{Basic properties}
This polynomial set has typical properties for an orthogonal set. However some ambiguity has appeared as some authors have introduced different naming conventions; we adhere to Zernike's protocol which is based upon the Euler identity.
 
\begin{enumerate}
\item Indexing: the order is given by $n=0,1,2,\dots$. The angular frequency is given by the $m$ which has the same parity as $n$ with $m\le n$. For example, for $n=3$ the two complex polynomials are referenced by $\lst{n,m} = \lst{(3,3),(3,1)}$.
%
\item The disc polynomials are separable in polar coordinates:
$$V_n^m(r,\theta) = R_n^{m}(r)e^{i m \theta}, \quad \lst{\disc(r,\theta):0\, \le \, r \, \le \, 1, \ 0 \, \le \, \theta \, < \, 2 \pi}$$
%
\item The polynomials are complex-valued functions:
$$R_n^{m}(r) \in \real{}, \ e^{i m \theta} \in \cmplx{}$$
%
\item The radial polynomials are standardized by
$$R_n^{m}(1) = 1$$
%
\item Using Euler's convention $e^{i \theta} = \cos \theta + i \sin \theta,$ the real and complex components are resolved in this manner: 
\begin{equation*}
  \begin{split}
    V_n^m(x_1,x_2) = R_n^m(r)e^{i m \theta} & = R_n^{m}(r)\paren{\cos m\theta + i \sin m\theta} \\
     & = Z_n^m(x_1,x_2) + i Z_n^{-m}(x_1,x_2)
  \end{split}
\end{equation*}
%
\item Use de Moivre's theorem to convert high frequency terms to multiple angle forms:
\begin{equation*}
 e^{i m \theta} = \cos m\theta + i \sin m\theta = \paren{\cos \theta + i \sin \theta}^m
\end{equation*}
%
\item Recursion relationship: let the midpoint $\omega=\frac{1}{2}(n-m)$ and the average $\sigma=\frac{1}{2}(n+m)$ then: 
\begin{equation}
  R_n^m(r) = \sum_{j=0}^{\omega}{\paren{-1}^j\frac{(n-j)!}{j!\paren{\omega-j}!\paren{\sigma-j}!}r^{n-2j}}
  \label{eq:recursion}
\end{equation}
%
\item Discrete and continuous rotational invariance: the angular frequency is quantized according to:
$$n-m \in 0, 2, 4, \dots$$
%
\item Orthogonality and the norm:
$$ \int\limits_{\overline{D}_2}{V_{n}^{m}(x_1,x_2) \overline{V}_{\nu}^{\mu}(x_1,x_2)\,dA} = \frac{\pi}{n+1}\,\delta_n^\nu \,\delta_m^\mu$$
%
\item Completeness: for any continuous surface $\psi(x_1,x_2)$ the error in the polynomial approximation can be arbitrarily small. Given $\eps>0$ there exists a set of complex numbers $a$ such that:
$$ \lim_{d\to\infty} \normi{\psi(x_1,x_2) - \sum_{n=0}^{d}{\sum_{m}{a_n^m V_n^m(x_1,x_2)}}} < \eps $$
However in finite precision computation there is a minimum $\eps$ before the Lagrange interpolating polynomial is reached.
%
\item Uniform approximation: The maximum residual error $\epsilon_{max}$ over the entire surface is bound by the contributions from the leading term of the remainder. For example a fit of degree $k$ where $k$ is odd:
$$\epsilon_{max} \, \le \, \abs{a_{k+1}^{0}} + \abs{a_{k+1}^{2}} + \dots + \abs{a_{k+1}^{k+1}}$$
%
\item Zero mean: the polynomials except $Z_0^0(x_1,x_2)$ have zero mean:
$$ \int\limits_{\overline{D}_2}{V_n^m(x_1,x_2)\,dA} = 0, \quad n>0; \qquad \int\limits_{\overline{D}_2}{V_0^0(x_1,x_2)\,dA} = 1$$
%
\item Parity: the polynomials have the same parity as their order $n$:
\begin{equation*}
  V_{n}^{m} (x,y) = \paren{-1}^{n}V_{n}^{m} (-x,-y); \qquad V_{n}^{m} \pzzbar = \paren{-1}^{n} V_{n}^{m} (-z,-\zbar).
\end{equation*}
In polar coordinates we note that cosine is an even function and sine is odd.
%
\end{enumerate}

Table \eqref{tab:Zernike in three coordinate systems} shows formulas for the lowest order terms in three different coordinate systems.
%%%%%%
\input{appendices/"app disk polynomials"/"tab Zernike polynomials"}


%%%%%%%%
\section{Rotational invariants}
\begin{equation}
  \psi \paren{r,\theta + \delta} = \psi \paren{r,\theta}, \qquad \delta > 0.
\end{equation}

\begin{table}[htdp]
\caption[Another look at some radial polynomials of Zernike]{Another look at some radial polynomials of Zernike. These radial polynomials are associated with the disk polynomials with angular velocity $m=0$ as in figure \eqref{fig:Zernike cutaways}.}
\begin{center}
\begin{tabular}{ccc}
%%
$(a)$ & \phantom{m} & $(b)$ \\[10pt]
%%
\includegraphics[ width = 2.5in ]{graphics/Zernike/"radials even"} &&
\includegraphics[ width = 2.5in ]{graphics/Zernike/"radials odd"} \\
%%
$R_{k}^{0}(r)$, $k=0,4,8,12$ &&
$R_{k}^{0}(r)$, $k=2,6,10,14$
%%
\end{tabular}
\end{center}
\label{fig:Zernike radials}
\end{table}%


%%%%%%%%%%%
%%%%%%%%%%%
\subsection{Gram-Schmidt process}
The radial polynomials with angular frequency $m=0$ can be constructed from a sequence of even monomial powers using the orthogonalization process named for Gram and Schmidt. For demonstration we will use the first three lowest even monomials and construct the first three lowest order Zernike radial polynomials:
\begin{equation*}
  \paren{1,r^{2},r^{4}} \quad \mapsto \quad \paren{R_{0}^{0}(r), R_{2}^{0}(r), R_{4}^{0}(r)}.
\end{equation*}
For this subset, $e^{im\theta} = e^{0} = 1$ and there is angular contribution;
\begin{equation}
  Z_{2k}^{0}(r,\theta) = R_{2k}^{0}(r), \qquad k = 0,1,2,\dots
\end{equation}
By inspection we see that these functions are rotationally-invariant.
\begin{equation}
  R_{2k}^{0} = \sum_{j=0}^{2k} { \paren{-1}^{j} \frac{(2k-j)!} {j! \paren{(k-j)!}^{2}} r^{2(k-j)}}
\label{eq:recursion}
\end{equation}

Begin by defining the inner product
\begin{equation*}
  \inner{f(r), g(r)} = \int_{0}^{1} f(r) g(r) r dr.
\end{equation*}
Note the geometrical factor of $r$ from the area element of polar coordinates. Using this inner product define the norm
\begin{equation*}
  \norm{f(r)} = \left< f(r), f(r) \right> = \int_{0}^{1} f^{2}(r) r dr.
\end{equation*}
Then constant term is
\begin{equation*}
  \tilde{R}_{0}^{0}(r) = \norm{1} = \frac{1} {2} \quad \to \quad R_{0}^{0}(r) = 1.
\end{equation*}
The quadratic term is constructed as
\begin{equation*}
  \tilde{R}_{2}^{0}(r) = r^{2} - \frac{ \left< r^{2}, R_{0}^{0}(r) \right> } {\norm{R_{0}^{0}(r)}} = r^{2} - \frac{1} {2} \quad \to \quad R_{2}^{0}(r) = 2r^{2} - 1.
\end{equation*}
The quartic term is constructed as
\begin{equation*}
  \tilde{R}_{4}^{0}(r) = r^{4} - \frac{ \left< r^{4}, R_{0}^{0}(r) \right> } {\norm{R_{0}^{0}(r)}} - \frac{ \left< r^{4}, R_{2}^{0}(r) \right> } {\norm{R_{2}^{0}(r)}} = r^{4} - r^{2} + \frac{1} {6} \quad \to \quad R_{4}^{0}(r) = 6r^{4} - 6r^{2} + 1.
\end{equation*}
%

%%%%%%%%%%%%%%%%%
This is alluded to by the Gram-Schmidt construction process.
Consider the sequence of even monomials defined by
\begin{equation}
  e_{k} (r) = r^{2k}, \qquad k=0,1,2\dots.
\end{equation}
To avoid the ambiguous form $0^{0}$ we make the definition
\begin{equation}
  e_{0}(r) \equiv 1.
\end{equation} 
\begin{myLemma}
Let $\mathcal{R}$ be the vector space spanned by the sequence of even monomials:
\begin{equation}
  \mathcal{R} = \spn{1, r^{2}, r^{4}, \dots, r^{k},\dots} \qquad k=1,2,3\dots
\end{equation}
This space is equivalent to the vector space of the rotationally-invariant disk polynomials over the unit disk. That is, for $0\le r\le 1$,
\begin{equation}
  \spn{e_{2k}(r)} = \spn{Z_{2k}^{0}(r)} \qquad k=1,2,3\dots
\end{equation}
\label{thm:Zernike spans}
\end{myLemma}

%%%%%%%%%%%%%%%%%
\begin{myTheorem}
The Zernike polynomial subset $Z_{2k}^{0}(r)$, $k=0,1,2,\dots$ is complete over the unit disk $\disc$ with respect to rotationally-invariant functions.
\label{thm:Zernike spans}
\end{myTheorem}

\begin{proof}
This is the polynomial sequence $1,r^{2},r^{4},\dots$.
Appeal to \ms
\end{proof}
%\input{chapters/"appendix fortran"/"sec organization"}


\endinput %-------------------------------