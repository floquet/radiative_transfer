\section{Titanium}

\begin{figure}[htbp] %  figure placement: here, top, bottom, or page
   \centering
   \includegraphics[ width = 3.75in ]{graphics/appendices/ti} 
   \caption[Locating titanium in the periodic table]{Locating titanium in the periodic table.}
   \label{fig:ti:wiki}
\end{figure}

Why titanium? Certainly being chemically inert is advantageous and nobel gases are even more inert than nitrogen. Argon was chosen because it could provide a distinct signal in the range of temperatures and pressures of interest.

%\input{chapters/"appendix fortran"/"sec organization"}
%\input{chapters/"appendix fortran"/"sec listings"}

shell structure for Ar: $1s^{2} \, 2s^{2}2p^{6} \, 3s^{2}3p^{6} 3d^{2} \, 4s^{2}$

[Ar]: $3d^{2} \, 4s^{2}$


\begin{figure}[htbp] %  figure placement: here, top, bottom, or page
   \centering
   \includegraphics[ width = 5.5in ]{graphics/appendices/"ti electronic structure"} 
%   \caption[Electronic structure for titanium]{Electronic structure for titanium. From \emph{The Periodic Table of Elements}, Macintosh, \verb=http://www.dhaatu.com/\verb=.}
   \caption[Electronic structure for titanium]{Electronic structure for titanium. From \emph{The Periodic Table of Elements}, Macintosh version 3.0.3.\footnote{\texttt{http://www.dhaatu.com/}}}
   \label{fig:ti:electronic structure}
\end{figure}

\begin{figure}[htbp] %  figure placement: here, top, bottom, or page
   \centering
   \includegraphics[ width = 5.5in ]{graphics/appendices/"ti general"} 
   \caption[Thermodynamic properties for titanium]{Thermodynamic properties for titanium. From \emph{The Periodic Table of Elements}, Macintosh v.3.0.3.\footnote{ibidem}}
   \label{fig:ti:thermodynamic properties}
\end{figure}


\endinput %-------------------------------