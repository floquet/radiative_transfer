\chapter{\label{ch:rte}Radiative transport equation}

The radiative transport equation describes how sentinel \xr s move through the plasma of the imploding DT gas to reach the detector. This is crucial for understanding how experimental signals are altered before they are measured.

The mathematical theory of radiative transport is rich and mature; characterized with impeccable scholarship. Important tomes are listed here.
\begin{itemize}
\item 1934, Eberhard Hopf, \emph{Mathematical problems of radiative equilibrium.}\cite{hopf1934}
\item 1952, Subrahmanyan Chandrasekhar, \emph{Radiative Transfer.}\cite{chandra1960}
\item 1960, Ida Busbridge, \emph{The mathematics of radiative transfer.}\cite{Busbridge1960}
\item 1963, Viktor Viktorovich Sobolev, \emph{A treatise on radiative transfer.}\cite{sobolev1963}
\item 2002, Annamaneni Peraiah, \emph{An Introduction to Radiative Transfer: Methods and Applications in Astrophysics.}\cite{Peraiah2002}
\end{itemize}
The goal of this dissertation is to apply and extend the insights from their general treatment to the specific problem before us. 

An equation so prominent in mathematical physics is certain to be expressed in different forms. Each form describes the same fundamental processes using different conventions.
\begin{enumerate}
\item $\alpha$: Absorption (or extinction) of photons
\item $j$: Creation (or generation) of photon
\item $\scat$: Scattering of photons into and out of the pencil.
\end{enumerate}
%%%
Specific intensity\index{intensity!specific}
%%%
\begin{equation}
  \td{I_{\nu}(s)}{s} + \alpha(s) I(s) = j(s) + \scat
  \label{eq:rte}
\end{equation}
The differential scatter cross section is $\csd$.

The actual measurement collects all the light along a line-of-sight. Effectively we are measuring $\mathcal{I}$, the \emph{integral} of the intensity\index{intensity!integrated} in equation \eqref{eq:rte}. 
Measurement
\begin{equation}
  \mathcal{I_{\nu}} = \int_{\gamma} I_{\nu} ds
  \label{eq:i:measured}
\end{equation}
The output quantity is the ratio of two spectral lines, typically the most intense.
\begin{equation}
  \boxed{
  \Xi_{\nu_{1}\nu_{2}}(\gamma) = \frac{\int_{\gamma} I_{\nu_{1}}ds} {\int_{\gamma} I_{\nu_{2}}ds}
  }
  \label{eq:xi}
\end{equation}

\begin{figure}[htbp] %  figure placement: here, top, bottom, or page
   \centering
   \includegraphics[ width = 3in ]{graphics/rte/"capsule limbs"} 
   \caption[The measured intensity is the integral along the line of sight]{The measured intensity is the integral along the line of sight. Two such limbs, $\gamma_{1}$ and $\gamma_{2}$, are shown here. The shading represents the value of some scalar field.}
   \label{fig:limbs}
\end{figure}

% <<<<<
\input{chapters/"ch rte"/"sec ode"}
\input{chapters/"ch rte"/"sec pde"}
\input{chapters/"ch rte"/"sec weak formulation"}
% <<<<<

\endinput %-------------------------------