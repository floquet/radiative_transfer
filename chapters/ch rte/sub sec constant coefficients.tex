\subsection{\label{subsec:constant coefficients}Constant coefficients}

Begin by treating the parameters $\alpha$ and $j$ as constants. This may seem unrelated to our problem, but when we study trajectories we will consider travel along isoclines where one of these parameters is constant.

%
\begin{landscape}
%
\begin{table}[htdp]
\caption[RTE with constant coefficients, cases I, II]{RTE with constant coefficients, cases I, II. An endpoint is set to unity.}
\begin{center}
\begin{tabular}{cc}
%
$I(s) = \frac{1}{\alpha}\paren{j+(\alpha-j)e^{\alpha(1-s)}}$ & $I(s) = \frac{1}{\alpha}\paren{j+(\alpha-j)e^{\alpha s}}$ \\[4pt]
%
$I(1) = 1$ & $I(0) = 1$ \\
%
\includegraphics[ width = 3.5in ]{graphics/rte/"constants 01 a"} &
\includegraphics[ width = 3.5in ]{graphics/rte/"constants 02 a"} \\[-12pt]
%
$\alpha = 2, 4, 6, 8, 10$& $j = 0, 2, 4, 6, 8, 10$\\
%
\includegraphics[ width = 3.5in ]{graphics/rte/"constants 01 j"} &
\includegraphics[ width = 3.5in ]{graphics/rte/"constants 02 j"} \\[-12pt]
%
$\alpha = 2, 4, 6, 8, 10$ & $j = 0, 2, 4, 6, 8, 10$\\
%
\end{tabular}
\end{center}
\label{tab:ode:constant:1}
\end{table}
%
%
\begin{table}[htdp]
\caption[RTE with constant coefficients, cases III, IV]{RTE with constant coefficients, cases III, IV. An endpoint is set to zero.}
\begin{center}
\begin{tabular}{cc}
%
$I(s) = \frac{j}{\alpha}\paren{1-e^{\alpha(1-s)}}$ & $I(s) = \frac{j}{\alpha}\paren{1-e^{\alpha s}}$ \\[4pt]
%
$I(1) = 0$ & $I(0) = 0$ \\
%
\includegraphics[ width = 3.5in ]{graphics/rte/"constants 03 a"} &
\includegraphics[ width = 3.5in ]{graphics/rte/"constants 04 a"} \\[-12pt]
%
$\alpha = 2, 4, 6, 8, 10$& $j = 0, 2, 4, 6, 8, 10$\\
%
\includegraphics[ width = 3.5in ]{graphics/rte/"constants 03 j"} &
\includegraphics[ width = 3.5in ]{graphics/rte/"constants 04 j"} \\[-12pt]
%
$\alpha = 2, 4, 6, 8, 10$ & $j = 0, 2, 4, 6, 8, 10$\\
%
\end{tabular}
\end{center}
\label{tab:ode:constant:2}
\end{table}
%%%%
\begin{table}[htdp]
\caption[RTE with constant coefficients, cases V, VI]{RTE with constant coefficients, cases V, VI. An endpoint derivative is set to zero.}
\begin{center}
\begin{tabular}{cc}
%
$I(s) = \frac{j}{\alpha}\paren{j-e^{\alpha(1-s)}}$ & $I(s) = \frac{j}{\alpha}\paren{j-e^{\alpha s}}$ \\[4pt]
%
$I'(1) = 1$ & $I'(1) = 1$ \\
%
\includegraphics[ width = 3.5in ]{graphics/rte/"constants 05 a"} &
\includegraphics[ width = 3.5in ]{graphics/rte/"constants 06 a"} \\[-12pt]
%
$\alpha = 2, 4, 6, 8, 10$& $j = 0, 2, 4, 6, 8, 10$\\
%
\includegraphics[ width = 3.5in ]{graphics/rte/"constants 05 j"} &
\includegraphics[ width = 3.5in ]{graphics/rte/"constants 06 j"} \\[-12pt]
%
$\alpha = 2, 4, 6, 8, 10$ & $j = 0, 2, 4, 6, 8, 10$\\
%
\end{tabular}
\end{center}
\label{tab:ode:constant:3}
\end{table}
%%%%
\begin{table}[htdp]
\caption[RTE with constant coefficients, cases VII, VIII]{RTE with constant coefficients, cases VII, VIII. An endpoint derivative is set to zero.}
\begin{center}
\begin{tabular}{cc}
%
$I(s) = \frac{j}{\alpha}$ & $I(s) = \frac{j}{\alpha}$ \\[4pt]
%
$I'(1) = 0$ & $I'(1) = 0$ \\
%
\includegraphics[ width = 3.5in ]{graphics/rte/"constants 07 a"} &
\includegraphics[ width = 3.5in ]{graphics/rte/"constants 07 a"} \\[-12pt]
%
$\alpha = 2, 4, 6, 8, 10$& $j = 0, 2, 4, 6, 8, 10$\\
%
\includegraphics[ width = 3.5in ]{graphics/rte/"constants 07 j"} &
\includegraphics[ width = 3.5in ]{graphics/rte/"constants 07 j"} \\[-12pt]
%
$\alpha = 2, 4, 6, 8, 10$ & $j = 0, 2, 4, 6, 8, 10$\\
%
\end{tabular}
\end{center}
\label{tab:ode:constant:4}
\end{table}
%
\end{landscape}
%

\begin{equation}
  \td{I(s)}{s} = -\alpha I(s)
\end{equation}
which we may integrate directly to produce
\begin{equation*}
  \fln{I(s)} = -\alpha \intbounds{ s }{ s_{0} }
\end{equation*}
or
\begin{equation}
  I(s) = I(0) e^{-\alpha s}
\end{equation}
for problems where the intensity at the origin is known.

% <<<<<
%\input{chapters/"ch XXX"/"subsec "}
% <<<<<

\endinput %-------------------------------