\chapter{Introduction}

This work describes the mathematical processes of interest in modeling the forward problem of radiative implosion of a small capsule containing a blend of Deuterium and Tritium (D-T) gases at the National Ignition Facility (NIF) at Lawrence Livermore National Laboratory (LLNL) in Livermore California.

In essence we are using the x-rays created during a shot on a D-T capsule to infer the temperature of the Ti trace gas to probe the behavior and interaction of the Deuterium and Tritium material under radiative compression. The principle mathematical tool is the radiative transport equation: the conditions for optically thin material are studied analytically; the \cite[MS]{Trent1981} condition of optically thick material is attacked via Monte Carlo simulation of individual \cite{Stefanov2009} photons traversing the capsule.\cite[p. 10]{chandra1960}

%%%%%%
\input{chapters/"ch 01"/"sec Purpose"}
\input{chapters/"ch 01"/"sec Experiment"}
\input{chapters/"ch 01"/"sec Measurement"}
%%%%%%


\endinput