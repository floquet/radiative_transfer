\section{Purpose}

Our purpose is to explore the mathematics of radiative transfer. To do justice to so rich a subject, we will trade breadth of coverage for depth and restrict the contact to inertial confinement fusion (ICF), a topic of intense research and interest. The physical background is based upon experiments performed at the National Ignition Facility (NIF). A golden hohlraum suspends a tent containing a target capsule filled with a mixture of deuterium and tritium (D-T) gas at low temperature. In fact, the temperature allows formation of an outer shell of D-T ice. This fuel is encased in a polycarbonate (CH) sphere. The final layer is a titanium ablater.

The experiment begins when large banks of powerful lasers fire in unison, focused on the interior of the hohlraum. As the gold is vaporized, the interior hohlraum walls ablate away, burning in the manner of rocket fuel. An intense barrage of energetic electrons and x-rays explode forth, shocking the capsule with a compressive force. This causes the W ablator to heat and undergo a rocket fuel burn. While the energy transmitted from the lasers to the capsule is not enough to ignite thermonuclear burn throughout the capsule, the goal is to compress and heat the fuel enough to create a central hotspot of D-T fusion which would explode outward and ignite (...) material. 

For this research, the most important component is a trace amount (0.001\%) of a sentinel gas, here argon. The electronic transitions of the almost completely ionized argon are what the detectors record. As the D-T plasma heats and compresses, the electrons in the argon gas are scoured off. The most tightly bound electronic shell, the innermost, provides the clearest experimental signal.

Even with the restriction of attention to ICF, the topic of radiative transfer is overly broad and we are compelled to further constrain the investigation. To do so, we will simply posit more specific conditions to explore two well known hydrodynamic instabilities. The Rayleigh-Taylor instability ... The Kelvin-Helmholtz instability ...

We rely heavily upon the astrophysics of collapsing stars. The subject has evolved rapidly and features complexities such as general relativity, anisotropies, and angular momentum which we shall set aside.   Fusion of the hydrogen isotopes represents a sliver of stellar nucleosynthesis a mature topic celebrated by the Nobel Prize with awards to Chandrasekhar and Fowler in 1983.

Practical mastery of radiation hydrodynamics dates from the late 1940's and is embodied in dozens of designs and hundreds of tests of thermonuclear weapons. While once guarded vigorously, a significant part of the science is now unclassified. For example, the ``boosting'' of fission devices whereby a sphere of fissile material collapses onto about a hotspot of D-T gas which ignites and boosts the fission burn.

So we begin an exploration with rigorous theory, established experiment, and rich toolkit of computational methods. Yet surprisingly the characterization of indirect drive ICF at NIF remains somewhat open. What happens inside the capsule after the holmium is irradiated? More precisely, what do we expect from the x-ray field of the argon? How will radiation move through the capsule to the detectors? This context fosters natural questions. What will the x-ray detectors see? How will hydrodynamic instabilities and thermodynamic processes manifest? 

% % % % % % % %
The engineering achievements at NIF are profound. The creation and marshaling of such intense laser power is a marvel. The capsules themselves are quite an accomplishment.


\endinput %-------------------------------