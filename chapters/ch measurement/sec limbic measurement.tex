\section{\label{sec:measurement:limbs}Limbic measurements}
The challenge is to interpret these measurements and provide some description of the scalar field being sampled.

%%%%%%%%%%%
\subsection{Detector geometry}
\begin{table}[htdp]
\caption[Areas of the region in the capsule sampled by the detectors]{Areas of the region in the capsule sampled by the detectors.}
\begin{center}
\begin{tabular}{ccccc}
%
 \raisebox{-0.5\height}{\includegraphics[ width = 1.5in ]{graphics/rte/"segments pie"}} &$-$&
 \raisebox{-0.5\height}{\includegraphics[ width = 1.5in ]{graphics/rte/"segments triangle"}} &$=$&
 \raisebox{-0.5\height}{\includegraphics[ width = 1.5in ]{graphics/rte/"segments limb"}} \\
%
 $\frac{1}{2} \arccos(y)$ & $-$ & $\frac{1}{2}y\sqrt{1-y^{2}}$ & $=$ & $\frac{1}{2}\paren{ \arccos(y) - y\sqrt{1-y^{2}} }$ 
%
\end{tabular}
\end{center}
\label{tab:measurement:basic area}
\end{table}%


\begin{figure}[htbp] %  figure placement: here, top, bottom, or page
   \centering
   \includegraphics[ width = 5.5in ]{\tildepath\bandpath/layout}
   \caption[Detectors record limbic samples]{The detectors record limbic samples.}
   \label{fig:detectors}
\end{figure}

%%%%0%%%%%%%%
\begin{table}[htdp]
\caption[X-ray detector geometry]{X-ray detector geometry. Note $A_{0}\doteq0$.}
\begin{center}
\begin{tabular}{cccccc}
%
 & $A_{k}$ &$=$ & $\int_{0}^{\sqrt{1 - x_{k}^{2}}} \int_{y_{k}}^{\sqrt{1 - x^{2}}} f(x,y) dy dx$ &$-$& $A_{k-1}$ \\[20pt]
%
 $k = 1$ &
 \raisebox{-0.45\height}{\includegraphics[ width = 1in ]{\tildepath\bandpath/"bands 01"}} & = &
 \raisebox{-0.45\height}{\includegraphics[ width = 1in ]{\tildepath\bandpath/"bases 01"}} & $-$ & \qquad 
 \raisebox{-0.45\height}{\includegraphics[ width = 1in ]{\tildepath\bandpath/"bands 00"}} \\[50pt]
%
 $k = 2$ &
 \raisebox{-0.45\height}{\includegraphics[ width = 1in ]{\tildepath\bandpath/"bands 02"}} & = &
 \raisebox{-0.45\height}{\includegraphics[ width = 1in ]{\tildepath\bandpath/"bases 02"}} & $-$ & \qquad 
 \raisebox{-0.45\height}{\includegraphics[ width = 1in ]{\tildepath\bandpath/"bands 01"}} \\[50pt]
%
 $k = 3$ &
 \raisebox{-0.45\height}{\includegraphics[ width = 1in ]{\tildepath\bandpath/"bands 03"}} & = &
 \raisebox{-0.45\height}{\includegraphics[ width = 1in ]{\tildepath\bandpath/"bases 03"}} & $-$ & \qquad 
 \raisebox{-0.45\height}{\includegraphics[ width = 1in ]{\tildepath\bandpath/"bases 02"}} \\[50pt]
%
 $k = 4$ &
 \raisebox{-0.45\height}{\includegraphics[ width = 1in ]{\tildepath\bandpath/"bands 04"}} & = &
 \raisebox{-0.45\height}{\includegraphics[ width = 1in ]{\tildepath\bandpath/"bases 04"}} & $-$ & \qquad 
 \raisebox{-0.45\height}{\includegraphics[ width = 1in ]{\tildepath\bandpath/"bases 02"}} \\[50pt]
%
 $k = 5$ &
 \raisebox{-0.45\height}{\includegraphics[ width = 1in ]{\tildepath\bandpath/"bands 05"}} & = &
 \raisebox{-0.45\height}{\includegraphics[ width = 1in ]{\tildepath\bandpath/"bases 05"}} & $-$ & \qquad 
 \raisebox{-0.45\height}{\includegraphics[ width = 1in ]{\tildepath\bandpath/"bases 04"}} 
%
\end{tabular}
\end{center}
\label{tab:detector geometry}
\end{table}
%
\clearpage

%%%%%%%%%%%
\subsection{Zones and sectors}
Zones are annular regions; sectors are horizontal partitions in the zones. The annulus is defined by inner and outer radii $\paren{ r_{i}, r_{o} }$. The sectors are defined by horizontal cuts at the top and bottom $\paren{ b, t }$. The quantities are ordered and because the capsule is mapped to the unit disk we may write
\begin{equation}
  \begin{split}
    0 \le b &< t \le 1, \\
    0 \le r_{i} &< r_{o} \le 1.
  \end{split}
\end{equation}

Let $n$ represent the number of detectors mapped to a quadrant of the capsule. The detectors are considered to have continuous coverage and there are no gaps between the detectors. This implies that each detector has a spatial extent\index{spatial extent} $\Delta$ given as
\begin{equation}
  \Delta = \frac{1}{n}, \qquad n \in 1,2,3\dots
  \label{eq:defn:delta}
\end{equation}
This is a convenient parameter to use when discussing partitions of the capsule interior.

The largest segment is shown in table \eqref{tab:measurement:basic area}. Using the assignments
\begin{equation}
  \begin{split}
    y &= 1 - \Delta, \\
    x &= \sqrt{1 - \paren{1-\Delta^{2}}} = \sqrt{\Delta (2-\Delta)}
  \end{split}
\end{equation}
\begin{equation}
  a_{11}(\Delta) = \half \paren{ \arccos (1-\Delta) - (1-\Delta) \sqrt{\Delta (2-\Delta)} }
\end{equation}
%
Observe that when there is only one partition, that is a single limb for the quarter capsule we get the expected answer
%
\begin{equation}
  a_{11}(1) = \frac{\pi}{4}.
\end{equation}
%
This reaffirms that a quarter of the unit disk has area $\pi/4$.

In figure \eqref{fig:tooth} we see an annular zone with inner radius $r_{i} = 5\Delta$ and outer radius $r_{o} = 6\Delta$. The horizontal cuts which are mapped to detector edges has top $t=3\Delta$ and bottom $b=2\Delta$. This domain defines a sector with area $a_{64}$. The index on the array corresponds to an address in a matrix and will be explored later.
%  figure placement: here, top, bottom, or page
\begin{figure}[htbp] 
   \centering
   \includegraphics[ width = 4in ]{graphics/rte/"zone area"} 
   \caption[Zones and sectors]{Basic zones and sectors which partition the capsule, the two basic domain types. The detectors correspond to the horizontal cuts and have spatial extent $\Delta$. We will see that these parallel cuts will create annular regions and the ultimate solution will be quantification of the values of a scalar field between $r_{i}$ and $r_{o}$. }
   \label{fig:tooth}
\end{figure}

We will see that the sectors can be grouped into three classifications depending upon the complexity of their domains.

\endinput %-------------------------------